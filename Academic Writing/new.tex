\documentclass[a4paper,UTF8,12pt]{ctexart}
\usepackage[T1]{fontenc}
\usepackage{amsfonts}
\usepackage{geometry}
\usepackage{booktabs}
\geometry{scale=0.8}
\usepackage{amsmath, amsthm, amssymb, graphicx}
\usepackage[bookmarks=true, colorlinks, citecolor=blue, linkcolor=black]{hyperref}
\linespread{1.5}
\setlength{\parskip}{1em}

    \title{浅析北京地铁近年站名翻译}
    \author{马逸飞\\522031910765\\shenjidf11@sjtu.edu.cn}
\begin{document}
    \maketitle
    \tableofcontents

\section{摘要}
    从 2018 年底开始直至 2022 年底,北京地铁对系统内全部或部分站名的翻译进行了五轮修订,从而产生了六套标准。每次翻译调整的规模或大或小,有的局限在一两条线路,有的则是全线网规模的修订。
    每次翻译标准的调整幅度也是大小不一的。
    整体而言,这样的调整每年进行一次,与新线开通同步进行(即每年十二月底)。

    本文通过分析这五次调整前后的六张地铁线网图,对这六种站名翻译标准进行整理归纳和分析评价;然后对一类站名的各种翻译版本纵向地整理比较,分析各版本的优劣;
    最后概括北京地铁站名翻译以及全国地名翻译存在的问题,并给出解决建议。
    
    \paragraph{关键词:} 地名\ 翻译\ 地铁\ 北京

\section{问题背景}
    为了更好地研究这个问题,我们需要补充一些背景材料。

    \subsection{地名的概念}
        地名是人们对各个地理实体赋予的专有名称。\cite{ref1}

        地名通常为两部分:专名和通名。专名通常是地名中用来区分各个地理实体的词。通名通常是地名中用来区分地理实体类别的词。\cite{ref1}
	    
        地理实体,一般指在现实世界中再也不能划分为同类现象的现象,如城市、自然村落、公共设施等。例如,城市可以继续划分,但是划分出的部分与其整体不属于同一类别。
    \subsection{地名翻译的主要争议}
        当前,我国地名翻译方式主要有两种:
        \begin{enumerate}
            \item 将全部地名都按照汉语拼音方案音译为英文;
            \item 将部分地名的所有专名和通名都音译;另一部分的通名意译,专名视情况处理。
        \end{enumerate}

        这两种方案都有相应的法律法规或国家标准支撑,均被大量采用。方案一的主要法理依据是《地名管理条例》(中华人民共和国国务院令 第753号):
        地名的罗马字母拼写以《汉语拼音方案》作为统一规范,按照国务院地名行政主管部门会同国务院有关部门制定的规则拼写。\cite{ref2}而对于“地名”的界定,该法规做出了如下规范:\\本条例所称地名包括:\\
        (一)自然地理实体名称;\\
        (二)行政区划名称;\\
        (三)村民委员会、居民委员会所在地名称;\\
        (四)城市公园、自然保护地名称;\\
        (五)街路巷名称;\\
        (六)具有重要地理方位意义的住宅区、楼宇名称;\\
        (七)具有重要地理方位意义的交通运输、水利、电力、通信、气象等设施名称;\\
        (八)具有重要地理方位意义的其他地理实体名称。\cite{ref3}\\

        方案二的主要法理依据是《关于改用汉语拼音作为我国人名地名罗马字母拼法的统一规范的报告》(国发[1978]192号文件):
        在各外语中地名的专名部分原则上音译,用汉语拼音字母拼写,通名部分(如省、市、自治区、江、河、湖、海等)采取意译。\cite{ref4}

        我们可以看出,方案一的法理依据是国务院令,并且在 2022 年重新发布并实施;而方案二的法理依据为 1978 年的文件,并且规格仅仅是国务院发布的通知。因此,方案一有更充分的法规支持。
        但是,这个国务院令并没有得到很好的执行。有许多城市的道路、公园等的英文名称采用了方案二,如上海的中山公园(Zhongshan Park)、南京西路(West Nanjing Road)等等。
    \subsection{地铁站名的翻译问题}
        国家标准《公共服务领域英文译写规范》中规定:地名的罗马字母拼写应符合我国语言文字和地名管理法律法规的规定。作为公共服务设施的台、站、港、场,以及名胜古迹、纪念地、游览地、企业事业单位等名称,根据对外交流和服务的需要,可以用英文对其含义予以解释。\cite{ref5}\\
        这项国家标准特别强调不与已有的地名管理法规冲突,但是,若地铁站以公共服务设施、游览地等命名,它的翻译应该采用哪项标准?当一个公共服务设施被用作地铁站名时,它也具有了重要地理方位意义,可以作为地名处理。

        各个拥有轨道交通的城市用实际翻译给出了答案:在截稿前,除天津、沈阳、西安、石家庄外,其余所有城市都采用了上一节中的方案二翻译地铁站名。
        
        而北京是一个极其特殊的例子:北京市在 2018 年前采用的翻译标准与《地名管理条例》和《汉语拼音正词法基本规则》不符,但通过后续的数次调整尝试在它们的框架下翻译地铁站名。最符合要求的是 2021 年底到 2022 年底之间的时期,
        然而,北京市在 2022 年底的调整将部分地铁站名翻译调整回了 2021 年的版本,与这两项法规(国家标准)偏离。
    \subsection{地名拼写中的分词连写问题}
        分词连写问题,通俗地说就是汉语拼音拼写过程中要不要在某处断开,以及在哪里断开的问题。如“人民路”的拼音应当写作 Renmin Lu。

        国家标准《汉语拼音正词法基本规则》中规定了汉语拼音中词的拼写规则,其中 6.2.2 小节对地名拼写的问题做了规范。\cite{ref6}

        第一条规定:汉语地名中的专名和通名,分写,每一分写部分的首字母大写。该条举出北京市(Beijing Shi)等例。

        之前提到的天津、沈阳等四座城市采用的处理方式是将站名拼音全部大写且不断词,如滨海国际机场 BINHAIGUOJIJICHANG。我们可以看出,这是不符合《汉语拼音正词法基本规则》规范的。
        
        那么,“南京西路”等地名中的“西”这类限定成分应当如何处理?6.2.2.2 条规定:专名与通名的附加成分,如是单音节的,与其相关部分连写。
        如“南京西路”应当翻译为 Nanjing Xilu。

        6.2.2.3条规定:已专名化的地名不再区分专名和通名,各音节连写。其下附有“酒仙桥(医院)”“王村(镇)”“黑龙江(省)”等例。但是这条规定十分模糊,主要问题在于如何界定“已专名化的地名”。只要是在另一个地名中当作专名的地名就是“已专名化的地名”吗?
        例如,“物资学院路”中的“物资学院”内部究竟应当连写还是分写?这项文件并没有明确的规定。在国家标准《中国地理实体通名汉语拼音字母拼写规则》中规定:专名化的通名是转化为专名组成部分的地名通名,\cite{ref7} 从而解决了这个问题。

        本条还规定:不需区分专名和通名的地名,各音节连写。《中国地理实体通名汉语拼音字母拼写规则》中细化了这一规定:自然村镇、聚落名称不区分地名专名和地名通名。下附示例“周口店”“五道口”等。
        
        那么,“地名 + 方位词”结构的地铁站名应当如何翻译?《汉语拼音正词法基本规则》中规定:名词与后面的方位词,分写。\cite{ref9} 此时地名应该被算作一个整体,即上文所说的“名词”。

\section{北京地铁历年站名翻译规则归纳与分析}
    \subsection{2018 年底第一次调整前的翻译规则}
        2009 年,北京市引入京港地铁运营 4 号线,从此确立了一直沿用到 2018 年底的翻译标准,从 4 号线推广到整个路网。

        翻译规则大致如下:
        \begin{itemize}
            \item 涉及汉语拼音的部分全部采用大写字母,绝大多数地名不断词,但“北京”二字的拼写采用 "Beijing"。(原因可能是仅首字母大写的 Beijing 已经被全世界熟知)
                \subitem 这个版本中少数地名会断词。以“所在地区名称 + 当地地名”命名的车站会断词,如通州北关 TONGZHOU BEIGUAN;
                东四十条 DONGSI SHITIAO 、和平里北街 HEPINGLI BEIJIE 会断词,但也有同类地名如安德里北街 ANDELIBEIJIE、丰台东大街 FENGTAIDONGDAJIE 不断词。断词标准存在少量争议,
                相关细节问题将在下一章归类讨论。
            \item 以城市功能性设施命名的站名会采用意译,其类别可以大致归纳如下:
                \subitem 车站,如北京站 Beijing Railway Station、北京西站 Beijing West Railway Station 等;
                \subitem 航站楼,如 2 号航站楼 Terminal 2;
                \subitem 各类场馆,如国家图书馆 National Library、中国美术馆 National Art Museum 等;
                \subitem 大学,如人民大学 RENMIN University、北京大学东门 East Gate of Peking University;
                \subitem 游览地或名胜古迹,如圆明园 YUANMINGYUAN Park、十三陵景区 Ming Tombs、雍和宫 YONGHEGONG Lama Temple 等;
                \subitem 功能性园区名称,如丰台科技园 FENGTAI Science Park、良乡大学城 LIANGXIANG University Town、生物医药基地 Biomedical Base 等。
            \item 方位词意译,附在地名之后,如四惠东 SIHUI East 等。
            \item 其余类别的站名一律用汉语拼音处理,与其他许多城市有最大区别的是道路名称不意译,如知春路 ZHICHUNLU 等。
        \end{itemize}
        
        这个版本的翻译最大的特点是:标准高度一致,易于归纳整理,且具有很强的可操作性。但是,它也有不符合《地名管理条例》与《汉语拼音正词法基本规则》之嫌:
        部分车站译名没有采用汉语拼音,违反《地名管理条例》有关规定;上述规则中许多站名翻译没有在专名与通名间断开,不符合《汉语拼音正词法基本规则》规定。

        这个版本的翻译也存在一些令人费解的问题,如“北京大学东门”“颐和园西门”都采用了意译,但是“北工大西门”“天坛东门”全部采用汉语拼音,其标准令人难以捉摸。
        以及“首经贸”站并没有像其他以大学命名的车站一样采用意译的方式,而是译为 SHOUJINGMAO。
        学者石乐认为采用拼音的方式翻译地名简称是可行的,但是对不熟悉中文的人士而言,将拼音联系到“首都经济贸易大学” (Capital University of Economics and Business) 仍有很大困难。\cite{ref10}

        该版本(以及后续所有版本)的翻译还有一大争议在于路名的翻译同道路标志牌(采用意译通名的方式)上不符,导致对汉语不熟悉的乘客无法将地铁车站与附近的道路相关联。\cite{ref11} 也有观点认为道路名可以作为指示所在街区的地名,可以不意译。
    \subsection{2018 年底调整后的翻译规则}
        2018 年底北京地铁开始大幅度调整站名翻译规则,但试行范围不大,仅为当时开通的新线路:6 号线西段和 8 号线南段。

        新旧标准间主要差异如下:
        \begin{itemize}
            \item 将专名与通名用空格断开,各部分首字母大写,其余字母改为小写,如杨庄 Yang Zhuang(关于杨庄是否应该断词的问题后续讨论);
            \item 方位词采用特殊的处理方式,用示例说明:大红门南 Dahongmen Nan (South)。
        \end{itemize}

        由于本次调整规模不大,许多类型的站名不在调整范围之列,故只能对新标准做出如上归纳。从中我们可以看出,北京正在为了符合《地名管理条例》与《汉语拼音正词法基本规则》而调整站名翻译规则。

        新规则的争议主要在于分词连写问题。如 6 号线西段的廖公庄、西黄村、杨庄三站都以当地自然村落命名,按《汉语拼音正词法基本规则》不应当断词,
        但是北京地铁翻译时将“庄”“村”等字均作为专名处理,有断词不当之嫌。\\
        又如 8 号线南段的车站“东高地”,译为 Donggao Di,但是“东高地”这一地名的命名过程是:该处西侧地势较低处已有地名“西洼地”,1958 年国防部五院一分院选址南苑时
        选址的领导将东侧的高地一同划为国防用地,从此“东高地”地名传开。\cite{ref12} 因此,“东高地”的正确断词方式应该是 Dong Gaodi,而不是当时采用的 Donggao Di。
    \subsection{2019 年底调整后的翻译规则}
        本次调整后的翻译规则与 2018 年底的较为相近,但也存在许多不同之处。本次调整范围是当时开通东延线的地铁 7 号线全线与开通南延线的地铁八通线全线。

        调整后的翻译规则存在模糊空间,暂归纳如下:
        \begin{itemize}
            \item 对于绝大多数地名,采用逐字断词的方式处理,如九龙山 Jiu Long Shan、郎辛庄 Lang Xin Zhuang。
                \subitem 特例一:同为自然村落名的南楼梓庄(命名来源:自然村落名“楼梓庄”,然而在朝阳区北侧有重名地名,为区分二者而将朝阳区南侧的命名为“南楼梓庄”)译为 Nan Louzi Zhuang。
                推测其原因是将“南”作为专名的限定成分处理,然后根据《汉语拼音正词法基本规则》将“南”与后面的部分断开,但这与本次调整中其他自然村落名的处理方式相抵触。
                \subitem 特例二:“万盛西”“万盛东”两站,“因万盛片区得名”,\cite{ref13} 但是翻译成 Wansheng Xi (W)。这与大红门南 Nan (South) 的处理方法不同,前者采用了简写而后者没有;同时,“万盛”中间也并没有断词,
                有标准不一之嫌。与此同时,“四惠东”翻译为 Si Hui Dong (E),在同一张地铁线网图上出现了有关方位词的三种标准。
                \subitem 特例三:同样在 7 号线上,“广渠门内”译作 Guangqu Men Nei,“广安门内”译作 Guang An Men Nei,这样的翻译十分令人费解,笔者难以分析其原因。
            \item 新规则下意译的站名有:传媒大学、欢乐谷景区、环球度假区、北京西站(在调整范围之列,但是翻译没有变化)。可以对应到 2018 年底之前的意译标准。
        \end{itemize}

        有一个值得注意的细节:北京地铁八通线有“管庄”站,15 号线有“关庄”站,这两站的读音仅能通过声调区别,但之前的翻译全都处理成 GUANZHUANG,可能导致混淆。
        本次调整采用的新规则参考了陕西省(Shaanxi Province)的处理方式 \footnote{推测这样的处理方式来自中华民国时期的国家标准:汉语罗马字。},将“管庄”译为 Guaan Zhuang,解决了这个问题。

        本次调整同样规模有限,但是我们已经可以看出:2018 年底与 2019 年底两次调整后的标准存在很大不同,并且 2019 年底调整内部就存在断词标准不一的问题。
        笔者认为,2019 年底的调整是北京地铁站名翻译乱象的开始,出现了数个难以解释的“多重标准”问题。
    \subsection{2020 年底调整后的翻译规则}
        本次调整后的规则同前两次都存在不同,主要差异在于分词连写问题的处理方式。本次调整范围为全线网。

        本次调整涉及众多车站,规则较为复杂,大致归纳如下:
        \begin{itemize}
            \item 有关何种地名意译的判别标准,维持 2018 年前的不变;部分翻译做出调整,主要是将语序调整至汉语语序以便对照,如北京大学东门从 East Gate of Peking University 调整至 Peking University East Gate。一并解决了天坛东门和北工大西门的历史遗留问题。
            \item 断词上,新规则相较前两版规则都变得更加保守,也即不断词的站名增加。断词的标准存在争议与模糊空间,可以根据线网图推测,是否断词的判定依据在于地名的通名
            (无论该地名是否是国家标准中规定不宜断词的自然村镇名),需要断词的通名大致归纳如下,细节问题将在后续章节讨论:
                \subitem 门(除北宫门);
                \subitem 口(包含其前面方位词),如五道口、西小口;
                \subitem 城,如巩华城;
                \subitem 府,如平西府(在昌平)、饶乐府(在房山);
                \subitem 园、苑(公园景区除外,它们意译),但是存在特例“果园”和“梨园”,“苹果园”正常断词。这一点很令人费解;
                \subitem 宫(作为著名景点的雍和宫除外,它采取意译),如旧宫、新宫;
                \subitem 街、路及其限定成分,如知春路、荣昌东街。
            \item “桥”的处理:该桥现存则断词,不复存在则不断词。\cite{ref14} 个人认为“桥”前是否断词的判断标准十分合理,值得推广。
            \item 方位词的处理:方位词前一律不断词,如“天通苑”按上述标准断词为 Tiantong Yuan,然而相邻的“天通苑南”站译为 Tiantongyuan Nan (S)。
            \item 复合式地名的处理:
                \subitem “所在地区名 + 当地地名”结构,两部分之间断词;
                \subitem 由地名作为专名的清华东路西口、惠新西街南口、惠新西街北口、物资学院路划分为三部分,如 Qinghua Donglu Xikou。
        \end{itemize}

        网络上也出现了一些探讨本次调整断词标准的文章。\cite{ref14} 整体而言,本次断词标准存在很大争议,存在标准不一与标准不客观的问题。\\
        本次调整也有很大的积极意义,使之前未被调整的、不符合《汉语拼音正词法基本规则》的全大写翻译(如 QINGHUADONGLUXIKOU 等)更加靠近国家规范。

    \subsection{2021 年底调整后的翻译规则}
        本次调整的规模同样是全线网。其内容主要是:按照《地名管理条例》要求,将所有车站的英文站名全部调整为汉语拼音,之前采取意译方式的,一部分车站不再意译,另一部分在括号中附注双站名。例如,国家图书馆 Guojia Tushuguan (National Library)。
        
        之前意译,现在不再意译的是功能性园区名,有生命科学园、丰台科技园、良乡大学城、生物医药基地(值得注意的是,这个站名拆分成了 Shengwu Yiyao Jidi)等,但是亦庄文化园 Yizhuang Wenhuayuan (Yizhuang Culture Park) 仍然意译,采用双站名处理。

        同时,将汉语拼音中的 "a" 全部调整为国际音标中的字母 "a"(\LaTeX 中无法打出这个字符,本文暂不做替换)。\cite{ref17} 对 a 进行的调整引起了极大的争议——《汉语拼音方案》中采用的也是拉丁字母 a,而国际音标 a 多用于教育界与日常书写。\cite{ref15} 这样的调整意义不大,并且导致整个地铁网络的几乎所有标志、指示牌、电子显示屏内容等均需要更换,
        徒增维护成本。而且有许多字体并没有设计国际音标 a(例如,这篇文章也存在这个问题),导致这样的更换并不能覆盖所有导视内容。

        之前提到过的“管庄”与“关庄”的问题,本次调整改用在拼音字母 a 上方加注声调的方式处理,摒弃了与 Shaanxi 类似的处理方式。笔者认为,声调符号在字体较小时难以辨识,且外国民众对此不熟悉,这样的处理方式劣于前一年的标准。
        
        新规则对分词连写问题作出了进一步调整,相较 2020 年底模棱两可的断词标准有了更为统一的规范,但这样的规范是否更加符合《汉语拼音正词法基本规则》还有待讨论。

        调整后的翻译规则将断词的范围进一步缩小,大致如下:
        \begin{itemize}
            \item 所有道路名称专名和通名之间均断词,如安德里北街 Andeli Beijie、知春路 Zhichun Lu。
            \item “桥”前面没有限定成分一律不断词,但“花乡东桥”“和平西桥”“东风北桥”间断词,如 Huaxiang Dongqiao。
            \item 本次调整后仍然需要意译的站名,其拼音一律断词。
            \item 复合式地名的处理:
                \subitem “所在地区名 + 当地地名”结构,两部分之间断词;
                \subitem 由地名作为专名的北京大学东门(附注意译的双站名)、森林公园南门(同上)、物资学院路等仍然分为三部分,如 Beijing Daxue Dongmen;
        \end{itemize}

        特别地,含有方位词的站名几乎不断词,并且均不附注英文 N/S/E/W,如大屯路东 Datunludong(与知春路 Zhichun Lu 对照)。
        特例为良乡大学城北和良乡大学城西,处理为 Liangxiang Daxuechengbei(后者同理)。

        在本次站名翻译调整的同时,还调整了表示“地铁站”的“站”字的翻译。例如,魏公村站出入口上方的标识牌上内容是:
        \begin{table}[ht]
            \centering
            \begin{tabular}{|c|}
            \hline
            \textbf{\begin{tabular}[c]{@{}c@{}}地铁魏公村站\\ Weigongcun Zhan\end{tabular}} \\ \hline
            \end{tabular}
        \end{table}
        \newpage
        (注:下方英文原本是 Weigongcun Station。)

        本次调整后的翻译标准较为统一,模糊空间比较小,这一点与 2018 年底前的稳定版本类似。本次调整由于大规模采用拼音在互联网上引发了激烈的争论。\cite{ref16} 
        有人将本次调整视为“文化自信”的表现,认为日本的公交站名也出现了很多 eki-mae(即日语中表示“站前”的词语的罗马音),中国广泛采用拼音甚至将“站”处理为 Zhan 未尝不可;
        同时,所有站点均标示拼音,也方便不熟悉中文的旅客问路(考虑到中国许多民众英语程度并不足以听懂英语地铁站名);
        也有人认为新标准会导致外国旅客很难将拼音 Zhan 与车站相联系,造成不便。而对于一些功能性场所,北京地铁采用了双站名的方式翻译,既考虑到合规性问题与翻译的表音性,
        也照顾到翻译的表意性。

    \subsection{2022 年底的调整}
        笔者认为,本次调整使得北京市地铁站名翻译的乱象达到了最高峰。本次调整 \textbf{部分}\ 恢复了上次调整前的翻译,而具体哪些站名恢复到之前的版本,很难做出简单的归纳。
        因此,本次调整可以说是极为混乱的。唯一可以确定的两点是,新版的导视中“站”不再处理为 "Zhan",而是恢复了 Station;以及北京地铁一年来努力调整的 "a" 重新替换为了英文字母 a。

        笔者曾经希望按照线路不同对翻译标准进行归类,但是在考察 1 号线时就已经失败——天安门西、天安门东采用的是 2021 年底调整的版本,即 Tian'anmenxi, Tian'anmendong;而四惠东译为 Sihui Dong (E),回归了 2020 年底的标准。
        8 号线的乱象更为令人费解:“奥林匹克公园”站采用 2021 年底的双站名,然而北侧“西小口”站恢复了 2020 年底的断词标准 Xixiao Kou;南段“珠市口”采用 2021 年底的标准 Zhushikou,而同属南段的“大红门南”却翻译成 Dahongmen Nan (S),这是 2020 年底的标准。
        研究至此,对本次翻译调整的归纳可以说彻底破产。

        总而言之,本次调整用“乱象”形容毫不为过。

\section{各类站名翻译的变化及其分析}
    本章以各种站名类别为主线,纵向梳理各类站名近年来的不同翻译规则,并简要分析各种规则的优劣,提出建议。

    由于 2018 年底、2019 年底的调整规模不大,相应规则的有关问题已经在上一章探讨过,本章突出 2018 年底前 $\rightarrow$ 2020 年底后 $\rightarrow$ 2021 年底后 $\rightarrow$ 2022 年底后这一主轴。
    \subsection{以方位词结尾的站名}
        有代表性的站名:永定门外(北京特色,用城门内外标示方位)、大红门南(便于与大红门站相对照)。

        各种规则可大致归纳如下:

        \begin{table}[h]
            \begin{tabular}{|l|l|l|l|}
            \hline
            \textbf{采用标准}     & \textbf{永定门外}            & \textbf{大红门南}              & \textbf{大红门(对照)}    \\ \hline
            2018 年底前 & YONGDINGMENWAI  & DAHONGMEN South\footnote{当时该站不存在,根据当时标准推测}   & DAHONGMEN  \\ \hline
            2020 年底后 & Yongdingmen Wai & Dahongmen Nan (S) & Dahong Men \\ \hline
            2021 年底后 & Yongdingmenwai  & Dahongmennan      & Dahongmen \\ \hline
            \end{tabular}
        \end{table}

        笔者认为,永定门外 2020 年底的处理方式是最为合理且符合规范的。我们之前已经提到,根据国家标准,方位词前应当断词;而且“永定门”作为一个地名,应当保持其完整性。
        而大红门南的处理方式都有待优化。第一个版本主要问题在于不合规范,第二个版本中的 "(S)" 会给外国游客带来困扰,很难让人第一时间联想到这是 South 的简写。而且外国旅客可能不知道 Nan 的含义,误认为 Dahongmen Nan 是一个地名,而后面的 (S) 表示方位,从而引起不必要的误会。
        第三种处理方式问题在于方位词前没有断词,不合规范,而且如此长且不断词的拼音很难让外国访客将它与 Dahongmen 相关联。更好的做法是参考需意译站名的双站名模式,改为 Dahongmen Nan (Dahongmen South),符合规范且方便理解。

    \subsection{复合式地名}
        本节主要讨论分词连写问题。有代表性的站名:惠新西街南口、清华东路西口、北京大学东门(前三者为完整地名作为专名的地名)、通州北苑(所在地区名 + 当地地名)。
        \begin{table}[h]
            \begin{tabular}{|l|l|l|l|}
            \hline
            \textbf{采用标准}     & \textbf{惠新西街南口}              & \textbf{清华东路西口}               & \textbf{通州北苑}             \\ \hline
            2018 年底前 & HUIXINXIJIE NANKOU  & QINGHUADONGLUXIKOU   & TONGZHOU BEIYUAN \\ \hline
            2020 年底后 & Huixin Xijie Nankou & Qinghua Donglu Xikou & Tongzhou Beiyuan \\ \hline
            2021 年底后 & Huixin Xijie Nankou & Qinghua Donglu Xikou & Tongzhou Beiyuan \\ \hline
            \end{tabular}
        \end{table}

        有关“惠新西街南口”“清华东路西口”的细节问题:我们从表中可以看到 2018 年底前两者的翻译标准存在差异。事实上,在北京地铁系统内的不同场合,“惠新西街南口”是否断词就存在差异。例如当时 10 号线的车载电视
        在滚动播放到站信息时“惠新西街南口”就没有断词,但是官方线网图上则断了词。

        对于前两个站名,如果将它们做出一次专名与通名的划分,那么结果一定是“惠新西街 / 南口”“清华东路 / 西口”,但是“惠新西街”“清华东路”本身是路名,也可以再做一次专名与通名的划分。关于这个问题,
        我们在“专名化的通名”部分讨论过,按照规定不应该在“惠新西街”中间断词,但是这样可能会破坏 "Huixin Xijie" 作为一个路名的整体性。也就是说,“惠新西街”单独出现时译作 Huixin Xijie,它在路口中出现
        就要译作 Huixinxijie,但该路口中的“惠新西街”指代的就是这条路。这样的标准合理性有待考察。笔者认为,北京市的处理方式是更加合理的。“北京大学东门”也是同理。

        而对于通州北苑这类“所在地区名 + 当地地名”结构的站名,三种标准难得地达成了一致,就是断词。这样的规定也十分合理,因为“通州北苑”本身并不是一个地名,而是地名加上限定的结构,处理时应当断开。
    \newpage
        
    \subsection{功能性场所、场馆、景区、园区等曾被意译的站名}
        有代表性的站名:北京西站、北京大学东门、丰台科技园。
        \begin{table}[ht]
            \begin{tabular}{|l|l|l|l|}
            \hline
            \textbf{采用标准} & \textbf{北京西站}                                                                           & \textbf{北京大学东门}                                                                          & \textbf{丰台科技园}       \\ \hline
            2018 年底前      & Beijing West Railway Station                                                            & \begin{tabular}[c]{@{}l@{}}East Gate\\ of Peking University\end{tabular}                 & FENGTAI Science Park \\ \hline
            2020 年底后      & Beijing West Railway Station                                                            & Peking Univ. East Gate                                                                   & Fengtai Science Park \\ \hline
            2021 年底后      & \begin{tabular}[c]{@{}l@{}}Beijingxi Zhan\\ (Beijing West Railway Station)\end{tabular} & \begin{tabular}[c]{@{}l@{}}Beijing Daxue Dongmen\\ (Peking Univ. East Gate)\end{tabular} & Fengtai Kejiyuan     \\ \hline
            \end{tabular}
        \end{table}
        
        从细节上看:

        有关火车站译名的问题,中国国家铁路的规范是:地名与方位词间不断词。例如“北京西站”译为 Beijingxi Railway Station。短期来看,北京地铁在无法撼动国铁站名翻译规范的前提下,
        应当考虑将它的第二站名(英文意译)改为与国家铁路同步。这里我们不讨论国铁标准的合理性。

        有关翻译语序的问题,2020 年底与 2021 年底的标准强调了汉英语序的一致性,个人认为这并无必要。既然要在第二站名中采用意译,那么就应当符合更加常见的英文用法。
        East Gate of Peking University 相较 Peking Univ. East Gate 语法更加自然,建议更换为前者。\\
        同时,北京地铁还对 University, International 等较长单词使用缩写 Univ., Int'l 代替,减小英文站名的长度。

        2021 年底将城市功能性园区的翻译规则调整为拼音转写,笔者认为这也具有一定的合理性,因为“科技园”等可以看作当地街区的名称,外国访客在确定方位时不需要知道这个园区是科学园区还是企业总部基地,这一点与公园、车站等具有明显特征的功能性场所不同。

        整体而言,北京地铁双站名的处理方式是十分合理的,能够兼顾合规性、表音性以及表意性,同时方便旅客问路和理解站名含义。
        但是许多场合可能没有足够的空间采用双站名,比如较为紧凑的地铁线网图,导致英文站名字体较小,不便于查看。
        尽管如此,笔者认为双站名依然是现行规定下最好的选择。

    \subsection{其他类别的站名}
        这类站名众多,以下举出三例:
        \begin{table}[h]
            \begin{tabular}{|l|l|l|l|}
            \hline
            \textbf{采用标准}     & \textbf{五道口}       & \textbf{安华桥}        & \textbf{和平西桥}          \\ \hline
            2018 年底前 & WUDAOKOU  & ANHUAQIAO  & HEPINGXIQIAO  \\ \hline
            2020 年底后 & Wudao Kou & Anhua Qiao & Heping Xiqiao \\ \hline
            2021 年底后 & Wudaokou  & Anhuaqiao  & Heping Xiqiao \\ \hline
            \end{tabular}
        \end{table}

        本节主要问题在于如何确定断词标准。2020 年底的规则中按照通名不同确定是否断词十分不客观:以必须断词的“口”字为例,“五道口”是自然村镇名 \cite{ref8},而“惠新西街南口”是一个具体的路口,两者存在本质差别,然而在北京地铁当时“一刀切”断词的标准下
        无法做出区分。2021 年底的标准对断词十分严格,只有少数地名需要断词,争议空间较小,但也存在一些问题,如部分可以进行“专名-通名”划分的站名没有断词的问题,例如上方的“安华桥”(现存的立交桥)。但同为现存立交桥的“和平西桥”断了词,令人费解。

        笔者认为,当我们很难判断一个地铁站名是否是自然村镇名,从而决定其是否应该断词时,最为合理的断词标准是:地铁站名所指的地理实体是现存的,也即 2020 年底对“桥”的处理方式。
        因为断词意味着断开处后面的部分是一个通名,应当表示地理实体的类别。倘若站名所指的地理实体不复存在,如北京四九城中的各个城门,在“门”字前面断开意味着当地应当有一个属于“门”(城门)的地理实体,而这与现状是不符的。
        在这种情况下,“朝阳门”等地名可以被视为城门拆除后指代城门原址附近街区的名称,应当作为一个整体。而在上方的示例“安华桥”中,该地铁站附近存在一个属于“桥”的地理实体,从而应当断词。而北京地铁以道路命名的车站附近都有相应道路,因此应当断词。2020 年底与 2021 年底的标准都坚持了这一点。

        这样的标准可以做到相对客观,并且具有统一性,对这类地铁站名只需要考察其指代的地理实体是否现存即可确定翻译,并且尽量做到了符合《汉语拼音正词法基本规则》中有关分词连写问题的规定。

\section{总结与展望:北京地铁站名翻译何去何从?}
    北京地铁在近五年间不断调整站名翻译规则,笔者可以理解北京地铁不断优化翻译标准的初衷,但是我们也必须意识到,如此频繁的调整是极为混乱的。
    面对北京地铁站名翻译的乱象,尤其是 2022 年底调整后的混乱状况,笔者不禁想要对北京地铁提出质疑:翻译规则一年一次更换,甚至出现恢复更早版本的现象,当下 2020 年底的标准与 2021 年底的标准共存,
    这不仅会给不熟悉中文的乘客带来很大的困扰,并且要不断更换列车、车站中的线网图、闪灯图标志、LED 屏显示内容等等,极大增加了维护成本。这样大的工作量势必导致标准调整只能逐步进行,进而使得本次调整前后的导视标志同时存在,
    导致同一站名在不同位置的翻译存在差异,进一步令乘客感到困扰。
    
    由此可见,当前重要的工作是彻底终结北京地铁站名翻译的乱象。北京地铁应当确立具有高度一致性(考虑到站名的多样性,可以存在少量模糊空间)的站名翻译标准,并且统一贯彻落实,使得各个站名的翻译保持稳定,并且保证同一站名在各种导视信息上的翻译全部一致。
    
    实际上,北京地铁站名翻译存在的问题仅仅是全国地名翻译存在问题的缩影:《地名管理条例》相关规定在业界存在很大争议,没有得到较好落实,各地翻译标准各异;相同地名在不同场合出现时翻译不同,引发外国访客误解。
    我们希望国家能够确立具有准确性、客观性、通用性,以及一定的一致性的地名翻译标准,并且在全国得以落实,进而将全国的地名翻译确定下来。

    在上文中,我们探讨了双站名中汉语拼音转写存在的理由之一——方便访客问路。为何外国旅客有如此大的问路需求?其中最主要的原因在于北京地铁站名翻译乃至中国地名翻译没有实现信息化、数据化。
    在当下,我们中国民众的出行可以做到几乎完全依赖功能强大的地图软件(或称导航软件),在软件功能不出现问题的情况下可以覆盖向路人问路的所有作用。
    但是,北京地铁官方 APP 没有英文支持,不熟悉中文的旅客只能够拿着纸质线网图、查看车站与车厢中的英文导视信息查找路线,遇到困难只好询问工作人员或路人;
    从全国的角度看,支持英文的谷歌地图、必应地图数据与现实情况严重脱节,地图信息久未维护;国内地图软件的龙头百度地图、高德地图没有英文模式,不利外国访客使用。

    我们希望北京地铁站名翻译乃至全国的地名翻译能够实现信息化,并且导入电子地图软件中,让对汉语不熟悉的访客更方便地在中国出行、生活。
    全国地名、道路名翻译的确定与信息化是一项浩大的工程。倘若这项工程无法彻底完成,我们可以引入计算机能够实现的翻译标准,对未录入翻译的地名统一批量处理,如采用《汉语拼音方案》一律音译处理等。
    
    希望北京地铁站名翻译乱象尽快得到解决,站名翻译实现信息化、数据化;希望全国地名翻译标准问题能够解决,并且将全国的地名翻译录入数据库中,让不熟悉中文的访客也能在中国更便捷地出行。

    祝愿这一天尽早到来。

\begin{thebibliography}{100}

\bibitem{ref1}中华人民共和国国家质量监督检验检疫总局、中国国家标准化管理委员会.\ 外语地名汉字译写导则\ 英语: GB/T 17693.1-2008.\ 2008.\ 第二章
\bibitem{ref2}中华人民共和国国务院.\ 地名管理条例(中华人民共和国国务院令 第753号).\ 2022.\ 第十五条
\bibitem{ref3}中华人民共和国国务院.\ 地名管理条例(中华人民共和国国务院令 第753号).\ 2022.\ 第三条
\bibitem{ref4}中国文字改革委员会、外交部、国家测绘总局、中国地名委员会.\ 关于改用汉语拼音作为我国人名地名罗马字母拼法的统一规范的报告(国发[1978]192号文件).\ 1978.\ 实施说明第三条
\bibitem{ref5}中华人民共和国国家质量监督检验检疫总局、中国国家标准化管理委员会.\ 公共服务领域英文译写规范\ 第 1 部分:通则: GB/T 30240.1—2013.\ 2013.\ 4.1.3
\bibitem{ref6}中华人民共和国国家质量监督检验检疫总局、中国国家标准化管理委员会.\ 汉语拼音正词法基本规则\ GB/T 16159-2012.\ 2012.\ 6.2
\bibitem{ref7}国家市场监督管理总局、中国国家标准化管理委员会.\ 中国地理实体通名汉语拼音字母拼写规则\ GB/T 38207—2019.\ 2019.\ 3.6
\bibitem{ref8}国家市场监督管理总局、中国国家标准化管理委员会.\ 中国地理实体通名汉语拼音字母拼写规则\ GB/T 38207—2019.\ 2019.\ 5.3.3
\bibitem{ref9}中华人民共和国国家质量监督检验检疫总局、中国国家标准化管理委员会.\ 汉语拼音正词法基本规则\ GB/T 16159-2012.\ 2012.\ 6.1.1.1
\bibitem{ref10}石乐.\ 北京地铁站名英译探析.\ 考试与评价(大学英语教研版).\ 2014(06).\ 35-38.\ DOI:10.16830/j.cnki.22-1387/g4.2014.06.008.
\bibitem{ref11}和静、郭晏然.\ 北京地铁站名英译初探.\ 大学英语(学术版).\ 2015,12(02).\ 243-247.
\bibitem{ref12}中国运载火箭技术研究院新闻中心.\ “东高地”的由来.\ 中国运载火箭技术研究院官网.\ 2015.08.21.\ \href{http://www.calt.com/n488/n754/c4243/content.html}{http://www.calt.com/n488/n754/c4243/content.html}
\bibitem{ref13}北京市规划自然资源委.\ 北京地铁 7 号线东延沿线 9 座车站命名预案.\ 2019.05.\ \href{https://web.archive.org/web/20190515054111/http://mini.eastday.com/bdmip/190515132414666.html}{网址过长,可点此访问}
\bibitem{ref14}地铁规划建设观察 引自 hat600.\ 北京新版地铁站名英文译法问题整理.\ 2021.01.\ \href{https://mp.weixin.qq.com/s/0HDydNVBMMhmU4F_vtYloQ}{网址过长,可点此访问}
\bibitem{ref15}维基百科.\ 汉语拼音.\ \href{https://zh.wikipedia.org/wiki/%E6%B1%89%E8%AF%AD%E6%8B%BC%E9%9F%B3}{网址过长,可点此访问}
\bibitem{ref16}知乎.\ 如何看待北京地铁站名新版翻译用大量汉语拼音代替英文?你觉得哪种翻译更好?.\ \href{https://www.zhihu.com/question/507909755}{https://www.zhihu.com/question/507909755}
\bibitem{ref17}Lyt Design.\ 北京地铁站名翻译乱象.\ Metroman地铁通.\ 2023.01.10.\ \href{https://mp.weixin.qq.com/s/CJPA2Dn2_GV0ARfjM_Ue1Q}{网址过长,可点此访问}
\bibitem{ref18}Web archive 中北京地铁线网图的存档.\ \href{https://web.archive.org/web/20230000000000*/https://bjsubway.com/jpg.html}{网址过长,可点此访问}
\end{thebibliography}

\end{document}